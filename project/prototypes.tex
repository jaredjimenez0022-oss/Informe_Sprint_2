\chapter{Prototipos de interfaz}
En este capítulo se presentan los prototipos de interfaz desarrollados para el sistema \textbf{Komuness}, 
enfocándose en las funcionalidades implementadas y mejoradas durante el \textit{Sprint 2}. 
El diseño de la interfaz prioriza la usabilidad, experiencia de usuario y accesibilidad, considerando 
las necesidades específicas del sistema de monetización y perfiles premium implementados.

El enfoque de diseño mantiene los principios de diseño centrado en el usuario establecidos en el Sprint 1, 
implementando interfaces intuitivas que facilitan el proceso de pago y gestión de membresías. 
Se ha prestado especial atención a la retroalimentación visual en procesos críticos como transacciones 
y manejo de errores, garantizando transparencia y confianza para los usuarios.

\section{Prototipos del Sistema de Membresías Premium}

El sistema de membresías premium permite a los usuarios actualizar sus cuentas para obtener beneficios adicionales,
implementando un flujo de pago seguro mediante integración con PayPal.

\subsection{Interfaz de Selección de Planes}

\begin{figure}[H] % o [htbp] si no usas el paquete float
    \centering
    \includegraphics[width=0.8\textwidth]{project/images/interfaz1.PNG} % ruta y tamaño de la imagen
    \caption{Vista principal de selección de planes premium, donde los usuarios pueden comparar y elegir entre diferentes opciones de membresía.}
    \label{fig:premium-planes}
\end{figure}


Características principales:
\begin{itemize}
  \item Comparación visual clara entre plan mensual (\$4.00/mes) y anual (\$8.00/año).
  \item Badge destacado de descuento (33\% OFF) en plan anual.
  \item Lista de beneficios incluidos en la membresía premium.
  \item Diseño con gradientes atractivos (amarillo-dorado) para resaltar el valor premium.
  \item Indicador visual del plan seleccionado con animación.
\end{itemize}

Características de usabilidad:
\begin{itemize}
  \item Información de precios grande y legible.
  \item Comparación de ahorro (precio original vs. precio con descuento).
  \item Botón de retroceso para facilitar la navegación.
  \item Descripción clara de la periodicidad de facturación.
\end{itemize}

\subsection{Interfaz de Pago con PayPal}

La Figura~\ref{fig:premium-paypal} muestra la integración de botones de PayPal para completar el proceso de pago.

\begin{figure}[H] % o [htbp] si no usas el paquete float
    \centering
    \includegraphics[width=0.8\textwidth]{project/images/interfaz1.PNG} % ruta y tamaño de la imagen
    \caption{muestra la integración de botones de PayPal para completar el proceso de pago.}
    \label{fig:premium-planes}
\end{figure}


\begin{itemize}
  \item Integración nativa de PayPal Buttons SDK.
  \item Resumen del plan seleccionado antes del pago.
  \item Indicadores de seguridad (icono de escudo, mensaje de encriptación).
  \item Estados de carga con animaciones mientras se procesa el pago.
  \item Diseño responsive adaptado a dispositivos móviles.
\end{itemize}

\subsection{Sistema de Manejo de Errores de Pago}

\begin{figure}[H] % o [htbp] si no usas el paquete float
    \centering
    \includegraphics[width=0.8\textwidth]{project/images/pagoerror.PNG} % ruta y tamaño de la imagen
    \caption{muestra el sistema avanzado de retroalimentación para errores de pago.}
    \label{fig:premium-planes}
\end{figure}

Características del sistema de errores:
\begin{itemize}
  \item Mensajes de error específicos según el tipo de fallo (conexión, fondos insuficientes, tarjeta inválida).
  \item Animación de shake para llamar la atención sobre el error.
  \item Lista de sugerencias de acciones correctivas.
  \item Códigos de color diferenciados (rojo para errores, amarillo para advertencias).
\end{itemize}

\subsection{Confirmación de Pago Exitoso}

\begin{figure}[H] % o [htbp] si no usas el paquete float
    \centering
    \includegraphics[width=0.8\textwidth]{project/images/Interfaz4.png} % ruta y tamaño de la imagen
    \caption{ Vista de confirmación exitosa con mensaje de felicitaciones y redirección automática al perfi}
    \label{fig:premium-planes}
\end{figure}

\begin{itemize}
  \item Toast notification con animación y emoji celebratorio.
  \item Mensaje claro de confirmación de actualización a premium.
  \item Redirección automática al perfil de usuario.
\end{itemize}

\section{Prototipos del Sistema de Alertas de Límites}

El sistema de alertas notifica a los usuarios cuando alcanzan su límite de publicaciones y les ofrece la opción de actualizar a premium.

\subsection{Modal de Límite Alcanzado}


\begin{figure}[H] % o [htbp] si no usas el paquete float
    \centering
    \includegraphics[width=0.8\textwidth]{project/images/Interfaz5Ori.png} % ruta y tamaño de la imagen
    \caption{Modal de alerta mostrando límite alcanzado con barra de progreso y botón para actualizar a premium }
    \label{fig:premium-planes}
\end{figure}

Características:
\begin{itemize}
  \item Header con gradiente amarillo-dorado llamativo.
  \item Icono de estrella animado con pulso.
  \item Estadísticas claras (X de Y publicaciones utilizadas).
  \item Lista de beneficios premium destacados.
  \item Botón de acción directa Actualizar a Premium.
  \item Opción de cerrar para continuar sin actualizar.
\end{itemize}

Principios de UX aplicados:
\begin{itemize}
  \item Intervención no invasiva (modal con opción de cerrar).
  \item Propuesta de valor clara antes de solicitar pago.
  \item Animaciones sutiles para captar atención sin molestar.
  \item Call-to-action destacado visualmente.
\end{itemize}

\section{Prototipos del Sistema de Gestión “Acerca de”}

La sección \textbf{Acerca de} permite a los administradores gestionar la información institucional de Komuness CR.

\subsection{Vista Pública de la Sección}

La Figura~\ref{fig:acerca-vista} muestra la vista pública accesible para todos los visitantes.

\begin{figure}[H]
    \centering
    \includegraphics[width=\textwidth]{project/images/Interfaz6.png} % Ajusta la ruta de la imagen
    \caption{Vista completa de la sección Acerca de, panel de información y datos de contacto.}
    \label{fig:acerca-vista}
\end{figure}

\noindent
\textbf{Elementos visuales:}
\begin{itemize}
    \item Banner principal con título y subtítulo institucional.
    \item Sección “Quiénes Somos” 
    \item Iconografía colorida para cada sección del acordeón.
    \item Datos de contacto con enlaces clicables (teléfono, correo electrónico, redes sociales).
\end{itemize}

\subsection{Interfaz de Edición (Administrador)}

La Figura~\ref{fig:acerca-edit} muestra el panel de administración para editar contenido.

\begin{figure}[H]
    \centering
    \includegraphics[width=\textwidth]{project/images/Interfaz8.png} % Ajusta la ruta de la imagen
    \caption{Panel de edición con formularios para todas las secciones de Acerca de, sistema de carga de imágenes y botones de acción.}
    \label{fig:acerca-edit}
\end{figure}

\noindent
\textbf{Características administrativas:}
\begin{itemize}
    \item Formularios organizados por secciones temáticas.
    \item Textareas con auto-resize para textos largos.
    \item Sistema de carga de imágenes con previsualización instantánea.
    \item Límite de tamaño de archivos (proyectos/equipo).
    \item Botones para eliminar imágenes individuales.
    \item Gestión de información de donaciones y contactos.
    \item Botones de acción: guardar cambios o cancelar.
    \item Feedback visual durante el guardado (estado de carga).
\end{itemize}


\subsection{Sistema de Carrusel de Proyectos}

La Figura~\ref{fig:acerca-carousel} muestra el carrusel interactivo con modal de expansión.

[[INSERTAR IMAGEN: Carrusel de imágenes con botones de navegación y modal de visualización en pantalla completa]]

\begin{itemize}
  \item Navegación mediante flechas (anterior/siguiente).
  \item Indicadores de posición (puntos).
  \item Click en imagen para abrir modal en tamaño completo.
  \item Modal con fondo oscuro semitransparente.
  \item Botón de cerrar (X) visible en la esquina.
  \item Transiciones suaves entre imágenes.
\end{itemize}

\section{Prototipos de Correcciones Frontend}

\subsection{Corrección del Filtro de Búsqueda en Biblioteca}

\begin{figure}[H]
    \centering
    \includegraphics[width=\textwidth]{project/images/Interfaz10.png} % Ajusta la ruta de la imagen
    \caption{Campo de búsqueda en biblioteca con resultados filtrados correctamente y opción de limpiar búsqueda}
    \label{fig:acerca-edit}
\end{figure}

Mejoras implementadas:
\begin{itemize}
  \item Limpieza automática de resultados al borrar el texto de búsqueda.
  \item Indicador visual de filtro activo.
  \item Botón para limpiar búsqueda rápidamente.
  \item Feedback inmediato al escribir (búsqueda en tiempo real).
\end{itemize}

\subsection{Sistema de Límite de Tamaño de Archivos}


\begin{figure}[H]
    \centering
    \includegraphics[width=\textwidth]{project/images/interfaz11.png} % Ajusta la ruta de la imagen
    \caption{muestra mensajes de error por archivos excedidos}
    \label{fig:acerca-edit}
\end{figure}

\begin{itemize}
  \item Validación de tamaño antes de iniciar carga.
  \item Mensaje de error específico indicando límite de 200MB.
  \item Cancelación de carga si se detecta tamaño excedido.
\end{itemize}

\section{Prototipos de Perfil de Usuario con Indicadores Premium}

\subsection{Vista de Perfil Usuario Básico}

\begin{figure}[H]
    \centering
    \includegraphics[width=\textwidth]{project/images/Interfaz5.png} % Ajusta la ruta de la imagen
    \caption{Perfil de usuario mostrando badge "Básico", barra de progreso de límite de publicaciones y botón para actualizar a premium}
    \label{fig:acerca-edit}
\end{figure}

Elementos visuales:
\begin{itemize}
  \item Badge identificador del tipo de cuenta (color azul para básico).
  \item Barra de progreso mostrando publicaciones utilizadas vs. límite.
  \item Contador numérico de publicaciones disponibles.
  \item Botón destacado Actualizar a Premium" con gradiente amarillo.
  \item Advertencia visual cuando se acerca al límite (barra roja).
\end{itemize}



\subsection{Diseño Responsive en Dispositivos Móviles}

La Figura~\ref{fig:premium-responsive} muestra la adaptación del sistema de pago a móviles.

[[INSERTAR IMAGEN: Vista móvil del checkout premium con diseño adaptado y botones de PayPal optimizados para touch]]

Características responsive:
\begin{itemize}
  \item Touch targets grandes (mínimo 44x44px) para botones.
  \item Stack vertical de elementos en pantallas pequeñas.
  \item Texto escalable y legible en dispositivos móviles.
  \item Botones de PayPal adaptados al ancho de pantalla.
  \item Modales ocupando alto completo en dispositivos pequeños.
\end{itemize}

\subsection{Características Generales de Diseño y Usabilidad}

\begin{itemize}
  \item Esquema de colores consistente: amarillo/dorado para premium, azul para básico.
  \item Animaciones sutiles para feedback (shake en errores, pulse en éxitos, fade-in en modales).
  \item Estados de carga claros con spinners y mensajes descriptivos.
  \item Mensajes de error amigables y orientados a la solución.
  \item Confirmaciones visuales para acciones críticas (pago exitoso, límite alcanzado).
  \item Navegación intuitiva con breadcrumbs y botones de retroceso.
  \item Accesibilidad: contraste WCAG 2.1 AA, navegación por teclado, ARIA labels.
\end{itemize}