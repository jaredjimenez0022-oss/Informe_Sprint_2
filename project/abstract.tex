\begin{center}
\thispagestyle{empty}
\vspace{2cm}
\LARGE{\textbf{RESUMEN}}\\[1.0cm]
\end{center}
\thispagestyle{empty}
Este documento presenta el informe del Sprint 2 del proyecto de plataforma digital para la comunidad de Tejarcillos, Alajuelita, promovido por Komuness CR. Este segundo producto mínimo viable (MVP) se enfoca en establecer un modelo de sostenibilidad económica mediante la implementación de un sistema de usuarios premium y pagos con PayPal, mientras se corrigen fallos críticos identificados en la iteración anterior.

El Sprint 2 aborda la monetización de la plataforma mediante la integración de roles de usuario diferenciados (básico y premium), límites de publicaciones y un flujo completo de transacciones con PayPal. Las funcionalidades implementadas en esta iteración incluyen: sistema de validación de límites de publicaciones, integración con PayPal Sandbox para procesamiento de pagos, mecanismos de alertas para conversión a premium, sección ``Acerca de`` para mostrar información relevante de Komuness, y correcciones críticas en los módulos de biblioteca y recuperación de contraseñas.

Este segundo MVP establece las bases para la sostenibilidad económica del proyecto, permitiendo la generación de recursos que aseguren el mantenimiento y crecimiento continuo de la plataforma. La iniciativa consolida el empoderamiento juvenil y la inclusión tecnológica mediante un modelo mixto que mantiene el acceso gratuito para funcionalidades básicas mientras ofrece beneficios ampliados mediante membresías premium, fortaleciendo así el impacto comunitario a largo plazo.