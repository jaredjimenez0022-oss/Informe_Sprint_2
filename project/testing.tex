\chapter{Control de calidad}
Introduzca en esta sección la estrategia utilizada para el control de calidad del producto. Incluya los miembros del equipo que participaron del proceso de control de calidad. Si utilizó alguna herramienta de pruebas automatizadas, descríbala en esta seccción. Procure citar y referenciar la bibliografía según corresponda.

\section{Guiones de pruebas}
Indique cuántos guiones de pruebas se utilizaron para comprobar la calidad del producto. Cuáles fueron los criterios empleados para definir cada prueba.

\begin{longtable}{ | p{2cm} | p{3cm} | p{4cm} | p{4cm} | c |}
      \hline
      \textbf{Historias} & \textbf{Descripción} & \textbf{Resultado Esperado} & \textbf{Resultado Obtenido} & \textbf{Condición}\\
      \hline
      RF001 y RF003 & AAAA & BBBB & CCCC & \color{ForestGreen}PASÓ\\
      \hline
      RF002 y RF005 & AAAA & BBBB & CCCC & \color{Mahogany}FALLÓ\\
      \hline
      
      \caption{Descripción rápida del guión de pruebas}
      \label{TestScript1}
\end{longtable}

\begin{longtable}{ | p{2cm} | p{3cm} | p{4cm} | p{4cm} | c |}
      \hline
      \textbf{Historias} & \textbf{Descripción} & \textbf{Resultado Esperado} & \textbf{Resultado Obtenido} & \textbf{Condición}\\
      \hline
      RF001 y RF003 & AAAA & BBBB & CCCC & \color{ForestGreen}PASÓ\\
      \hline
      RF002 y RF005 & AAAA & BBBB & CCCC & \color{Mahogany}FALLÓ\\
      \hline
      
      \caption{Descripción rápida del otro guión de pruebas. Note que pueden haber varios guiones para la misma iteración.}
      \label{TestScript1}
\end{longtable}
