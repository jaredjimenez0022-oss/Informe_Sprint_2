\chapter{Control de calidad}

El proceso de control de calidad implementado en el proyecto \textbf{Komuness} durante el Sprint 2 se enfocó en garantizar la estabilidad, funcionalidad y seguridad del sistema de monetización mediante integración con PayPal, así como la corrección de errores críticos identificados en el Sprint 1.

El equipo adoptó un enfoque de \textit{testing} riguroso, donde cada funcionalidad crítica relacionada con pagos fue sometida a múltiples niveles de verificación antes de su integración al sistema principal. Se implementaron pruebas específicas para validar el sistema de reintentos, manejo de errores de red y actualización automática de perfiles de usuario.

La estrategia de control de calidad se estructuró en cuatro pilares fundamentales:
\begin{itemize}
    \item \textbf{Pruebas funcionales:} verificación de que cada requerimiento cumple con las especificaciones, incluyendo flujos de pago completos.
    \item \textbf{Pruebas de integración:} validación de la comunicación entre frontend, backend y servicios externos (PayPal).
    \item \textbf{Pruebas de seguridad:} verificación de autenticación, autorización y validación de webhooks de PayPal.
    \item \textbf{Pruebas de manejo de errores:} confirmación de comportamiento robusto ante fallos de red, timeouts y errores de pago.
\end{itemize}

\section{Equipo participante}
Los miembros involucrados en el proceso de control de calidad fueron:
\begin{itemize}
    \item \textbf{Fredrik Aburto Jiménez}
    \item \textbf{Angélica Díaz Barrios} 
    \item \textbf{Andrés Salas Araya} 
\end{itemize}

\section{Herramientas utilizadas}
Se emplearon diversas herramientas y entornos para garantizar la calidad:
\begin{itemize}
    \item \textbf{PayPal Sandbox:} entorno de pruebas para validar flujos de pago sin transacciones reales.
    \item \textbf{Jest + Supertest:} framework de pruebas unitarias y de integración para backend en Node.js.
    \item \textbf{React Testing Library:} pruebas de componentes frontend y flujos de usuario.
    \item \textbf{Postman:} pruebas manuales de endpoints API y validación de webhooks.
    \item \textbf{Chrome DevTools:} debugging de errores de red y validación de reintentos.
    \item \textbf{Navegadores múltiples:} validación cross-browser (Chrome, Firefox, Safari, Edge).
    \item \textbf{Dispositivos físicos:} validación en smartphones y tablets reales para checkout responsive.
\end{itemize}

\section{Guiones de pruebas}

Se desarrollaron \textbf{10 guiones de pruebas específicos} para validar las funcionalidades críticas implementadas durante el Sprint 2.  
Los criterios empleados fueron:
\begin{itemize}
    \item Cobertura funcional completa de sistema de pagos y membresías.
    \item Escenarios de error (fallos de red, pagos rechazados, timeouts).
    \item Validación de reintentos automáticos con backoff exponencial.
    \item Pruebas de seguridad (webhooks, autenticación, autorización).
    \item Compatibilidad cross-platform y responsive.
\end{itemize}

\subsection{Guión 1: Integración de PayPal - Configuración (RF014)}

\begin{longtable}{ | p{2cm} | p{4cm} | p{4cm} | p{4cm} | c |}
\hline
\textbf{Historias} & \textbf{Descripción} & \textbf{Resultado Esperado} & \textbf{Resultado Obtenido} & \textbf{Condición}\\
\hline
RF014 & Configuración de SDK de PayPal en entorno sandbox con credenciales de prueba & Variables de entorno configuradas, autenticación exitosa con PayPal API & Configuración exitosa, tokens de acceso generados correctamente & \color{ForestGreen}PASÓ \\
\hline
RF014 & Creación de orden de pago para plan mensual (\$4.00 USD) & Orden creada en PayPal con ID único y monto correcto & Orden creada correctamente, custom\_id con userId incluido & \color{ForestGreen}PASÓ \\
\hline
RF014 & Creación de orden de pago para plan anual (\$8.00 USD) & Orden creada en PayPal con ID único y monto correcto & Orden creada correctamente, descripción del plan incluida & \color{ForestGreen}PASÓ \\
\hline
\caption{Guión de pruebas 1: Configuración de PayPal SDK}
\label{TestScript1}
\end{longtable}

\subsection{Guión 2: Flujo Completo de Pago Exitoso (RF015, RF018)}

\begin{longtable}{ | p{2cm} | p{4cm} | p{4cm} | p{4cm} | c |}
\hline
\textbf{Historias} & \textbf{Descripción} & \textbf{Resultado Esperado} & \textbf{Resultado Obtenido} & \textbf{Condición}\\
\hline
RF015 y RF018 & Usuario básico selecciona plan mensual y completa pago con PayPal & Pago procesado, usuario actualizado a tipoUsuario=3 (premium), límite de publicaciones incrementado & Usuario actualizado correctamente, límite cambiado de 10 a 50, toast de éxito mostrado & \color{ForestGreen}PASÓ \\
\hline
RF015 y RF018 & Usuario básico selecciona plan anual y completa pago & Pago procesado, usuario actualizado a premium, descuento aplicado correctamente & Usuario premium, registro de pago guardado en BD con monto \$8.00 & \color{ForestGreen}PASÓ \\
\hline
RF015 & Validación de idempotencia en pagos duplicados & Si se recibe el mismo captureId dos veces, la segunda debe ser rechazada sin duplicar actualización & Sistema detectó duplicado, retornó idempotent:true, no se actualizó usuario dos veces & \color{ForestGreen}PASÓ \\
\hline
\caption{Guión de pruebas 2: Flujo de pago exitoso}
\label{TestScript2}
\end{longtable}

\subsection{Guión 3: Sistema de Reintentos y Manejo de Errores (RF016)}

\begin{longtable}{ | p{2cm} | p{4cm} | p{4cm} | p{4cm} | c |}
\hline
\textbf{Historias} & \textbf{Descripción} & \textbf{Resultado Esperado} & \textbf{Resultado Obtenido} & \textbf{Condición}\\
\hline
RF016 & Simulación de timeout en llamada a PayPal API & Sistema debe reintentar hasta 3 veces con backoff exponencial (1s, 3s, 9s) & 3 reintentos ejecutados correctamente, delays medidos: 1.02s, 3.01s, 9.03s & \color{ForestGreen}PASÓ \\
\hline
RF016 & Simulación de error de red (ECONNREFUSED) & Sistema debe marcar error como recuperable y reintentar & Error categorizado correctamente como CONNECTION\_ERROR, 3 reintentos realizados & \color{ForestGreen}PASÓ \\
\hline
RF016 & Simulación de fondos insuficientes & Sistema debe marcar error como NO recuperable y fallar sin reintentar & Error categorizado como INSUFFICIENT\_FUNDS, no se reintentó, mensaje claro al usuario & \color{ForestGreen}PASÓ \\
\hline
RF016 & Simulación de error 500 de PayPal & Sistema debe reintentar (error recuperable del servidor) & Error categorizado como PAYPAL\_SERVER\_ERROR, 3 reintentos ejecutados & \color{ForestGreen}PASÓ \\
\hline
RF016 & Historial de reintentos guardado en base de datos & Cada reintento debe registrarse con timestamp, attemptNumber, errorCode & Historial guardado correctamente en Payment.retryHistory con todos los campos & \color{ForestGreen}PASÓ \\
\hline
\caption{Guión de pruebas 3: Sistema de reintentos}
\label{TestScript3}
\end{longtable}

\subsection{Guión 4: Webhooks de PayPal (RF015)}

\begin{longtable}{ | p{2cm} | p{4cm} | p{4cm} | p{4cm} | c |}
\hline
\textbf{Historias} & \textbf{Descripción} & \textbf{Resultado Esperado} & \textbf{Resultado Obtenido} & \textbf{Condición}\\
\hline
RF015 & Recepción de webhook PAYMENT.CAPTURE.COMPLETED & Webhook recibido, firma verificada, usuario actualizado a premium & Webhook procesado correctamente, firma válida, usuario actualizado & \color{ForestGreen}PASÓ \\
\hline
RF015 & Recepción de webhook con firma inválida & Webhook debe ser rechazado por seguridad & Webhook rechazado, status 401, mensaje de error claro & \color{ForestGreen}PASÓ \\
\hline
RF015 & Recepción de webhook duplicado (mismo eventId) & Sistema debe detectar duplicado y no procesar dos veces & Duplicado detectado por índice único en eventId, retornó idempotent:true & \color{ForestGreen}PASÓ \\
\hline
\caption{Guión de pruebas 4: Webhooks de PayPal}
\label{TestScript4}
\end{longtable}

\subsection{Guión 5: Sistema de Límites de Publicaciones (RF013, RF018)}

\begin{longtable}{ | p{2cm} | p{4cm} | p{4cm} | p{4cm} | c |}
\hline
\textbf{Historias} & \textbf{Descripción} & \textbf{Resultado Esperado} & \textbf{Resultado Obtenido} & \textbf{Condición}\\
\hline
RF013 & Usuario básico con 9 publicaciones intenta crear la décima & Publicación creada exitosamente (dentro del límite de 10) & Publicación creada, contador actualizado a 10/10 & \color{ForestGreen}PASÓ \\
\hline
RF013 y RF018 & Usuario básico con 10 publicaciones intenta crear una más & Sistema debe rechazar con error 403 y mostrar alerta para actualizar a premium & Error 403 retornado, modal de alerta mostrado, botón "Actualizar a Premium" visible & \color{ForestGreen}PASÓ \\
\hline
RF013 & Usuario premium con límite de 50 publica exitosamente & Publicación creada sin restricciones hasta llegar a 50 & Publicaciones 1-50 creadas exitosamente & \color{ForestGreen}PASÓ \\
\hline
RF018 & Modal de alerta muestra progreso visual del límite & Barra de progreso refleja porcentaje usado (ej: 10/10 = 100\%) & Barra de progreso en 100\%, color rojo, mensaje de límite alcanzado & \color{ForestGreen}PASÓ \\
\hline
\caption{Guión de pruebas 5: Límites de publicaciones}
\label{TestScript5}
\end{longtable}

\subsection{Guión 6: Gestión de Configuración de Límites (RF013)}

\begin{longtable}{ | p{2cm} | p{4cm} | p{4cm} | p{4cm} | c |}
\hline
\textbf{Historias} & \textbf{Descripción} & \textbf{Resultado Esperado} & \textbf{Resultado Obtenido} & \textbf{Condición}\\
\hline
RF013 & Admin actualiza límite básico de 10 a 15 & Configuración guardada, usuarios básicos nuevos tienen límite 15 & Actualización exitosa, nuevos usuarios básicos con límite 15/15 & \color{ForestGreen}PASÓ \\
\hline
RF013 & Admin actualiza límite premium de 50 a 100 & Configuración guardada, usuarios premium con límite ampliado & Límite actualizado, usuarios premium existentes reflejan nuevo límite 100 & \color{ForestGreen}PASÓ \\
\hline
RF013 & Admin intenta configurar límite básico mayor que premium & Sistema debe rechazar con validación de error & Error retornado, mensaje: "límite premium debe ser mayor o igual al básico" & \color{ForestGreen}PASÓ \\
\hline
\caption{Guión de pruebas 6: Configuración de límites}
\label{TestScript6}
\end{longtable}

\subsection{Guión 7: Sección Acerca de (RF017)}

\begin{longtable}{ | p{2cm} | p{4cm} | p{4cm} | p{4cm} | c |}
\hline
\textbf{Historias} & \textbf{Descripción} & \textbf{Resultado Esperado} & \textbf{Resultado Obtenido} & \textbf{Condición}\\
\hline
RF017 & Admin crea/actualiza contenido de "Quiénes Somos" & Contenido guardado en BD, visible en vista pública & Texto guardado correctamente, mostrado en acordeón de vista pública & \color{ForestGreen}PASÓ \\
\hline
RF017 & Admin sube imagen a carrusel de proyectos & Imagen almacenada en servidor, URL guardada en BD, visible en carrusel & Imagen subida a /uploads, path guardado en imagenesProyectos array & \color{ForestGreen}PASÓ \\
\hline
RF017 & Admin intenta subir más de 10 imágenes a carrusel & Sistema debe rechazar con límite de 10 imágenes & Error 400, mensaje: "Máximo 10 imágenes para proyectos" & \color{ForestGreen}PASÓ \\
\hline
RF017 & Visitante no autenticado accede a sección "Acerca de" & Contenido visible sin requerir autenticación & Sección accesible públicamente, todos los campos mostrados & \color{ForestGreen}PASÓ \\
\hline
\caption{Guión de pruebas 7: Sección Acerca de}
\label{TestScript7}
\end{longtable}

\subsection{Guión 8: Correcciones de Errores del Sprint 1 (RF011, RF012)}

\begin{longtable}{ | p{2cm} | p{4cm} | p{4cm} | p{4cm} | c |}
\hline
\textbf{Historias} & \textbf{Descripción} & \textbf{Resultado Esperado} & \textbf{Resultado Obtenido} & \textbf{Condición}\\
\hline
RF011 & Usuario busca archivo en biblioteca y luego borra el texto & Resultados deben limpiarse, mostrando todos los archivos de nuevo & Filtro limpiado correctamente, vista resetea a estado inicial & \color{ForestGreen}PASÓ \\
\hline
RF011 & Usuario intenta subir archivo de 250MB a biblioteca & Sistema debe rechazar con error de límite (200MB máximo) & Error 413, mensaje: "archivo excede límite de 200MB" & \color{ForestGreen}PASÓ \\
\hline
RF012 & Usuario solicita recuperación de contraseña en producción & Email enviado exitosamente, sin error 504 Gateway Timeout & Email recibido, link de recuperación válido, timeout corregido & \color{ForestGreen}PASÓ \\
\hline
\caption{Guión de pruebas 8: Correcciones del Sprint 1}
\label{TestScript8}
\end{longtable}

\subsection{Guión 9: Sistema de Moderación Premium (RF019)}

\begin{longtable}{ | p{2cm} | p{4cm} | p{4cm} | p{4cm} | c |}
\hline
\textbf{Historias} & \textbf{Descripción} & \textbf{Resultado Esperado} & \textbf{Resultado Obtenido} & \textbf{Condición}\\
\hline
RF019 & Usuario premium edita su publicación existente & Publicación actualizada, admin notificado por correo & Edición guardada, correo enviado a admin con link a publicación & \color{ForestGreen}PASÓ \\
\hline
RF019 & Usuario premium elimina su publicación & Publicación marcada como eliminada, admin notificado & Publicación soft-deleted, notificación enviada correctamente & \color{ForestGreen}PASÓ \\
\hline
RF019 & Usuario básico intenta editar publicación & Sistema debe permitir edición sin restricción de tipo de usuario & Edición permitida para todos los usuarios autenticados & \color{ForestGreen}PASÓ \\
\hline
\caption{Guión de pruebas 9: Moderación premium}
\label{TestScript9}
\end{longtable}

\subsection{Guión 10: Pruebas de Seguridad y Autenticación (RNF)}

\begin{longtable}{ | p{2cm} | p{4cm} | p{4cm} | p{4cm} | c |}
\hline
\textbf{Historias} & \textbf{Descripción} & \textbf{Resultado Esperado} & \textbf{Resultado Obtenido} & \textbf{Condición}\\
\hline
RNF & Usuario no autenticado intenta acceder a /checkout-premium & Redirección a login con mensaje informativo & Middleware de autenticación funcionando, redirección a /iniciarSesion & \color{ForestGreen}PASÓ \\
\hline
RNF & Usuario básico intenta actualizar límites globales (endpoint admin) & Acceso denegado (error 403) & Middleware de roles funcionando, error 403 retornado & \color{ForestGreen}PASÓ \\
\hline
RNF & Usuario intenta realizar pago sin token JWT válido & Sistema debe rechazar la petición & Error 401, mensaje: "Token inválido o expirado" & \color{ForestGreen}PASÓ \\
\hline
\caption{Guión de pruebas 10: Seguridad y autenticación}
\label{TestScript10}
\end{longtable}