\chapter{Arquitectura del sistema}

\section{Diseño general del sistema}

La arquitectura del sistema desarrollado para Komuness mantiene el patrón Modelo-Vista-Controlador (MVC) establecido en el Sprint 1, con extensiones significativas para soportar el sistema de monetización y membresías premium implementado en el Sprint 2.

El diseño continúa basándose en una API RESTful siguiendo los principios de Fielding (2000), con nuevas capas de seguridad y procesamiento de pagos que mantienen la separación clara de responsabilidades y facilitan el mantenimiento del código.

\subsection*{Cambios arquitectónicos del Sprint 2}

Durante esta iteración se realizaron extensiones arquitectónicas significativas sin romper la compatibilidad con el sistema existente:

\begin{enumerate}
  \item Integración con servicios externos de pago: Se añadió capa de integración con PayPal SDK, incluyendo manejo de OAuth2, verificación criptográfica de webhooks y procesamiento de eventos asíncronos.
  
  \item Sistema de reintentos y resiliencia: Implementación de utilidades de reintento con backoff exponencial (paymentRetry.ts) y manejador centralizado de errores de pago (paymentErrorHandler.ts), mejorando la confiabilidad del sistema ante fallos temporales de red.
  
  \item Middleware de límites de publicaciones: Nuevo middleware de validación que intercepta creación de publicaciones y verifica límites según tipo de usuario, consultando configuración dinámica en base de datos.
  
  \item Sistema de moderación con versionado: Almacenamiento temporal de cambios propuestos mediante campo pendingUpdate en modelo de publicaciones, permitiendo flujo de aprobación sin afectar contenido visible.
  
  \item Colección de configuración: Nueva entidad Configuracion que permite ajustar parámetros del sistema (límites de publicaciones) sin redespliegue, consultable mediante API.
  
  \item Auditoría de pagos: Colección payments para registro completo de transacciones, incluyendo captureId, eventId, historial de reintentos y metadata para análisis y soporte.
\end{enumerate}

\subsection*{Arquitectura Backend}

El backend mantiene Express.js con TypeScript, reforzando el tipado estático con nuevas interfaces para entidades de pago y configuración:

\begin{itemize}
  \item Controladores: Se añadieron controladores especializados:
  \begin{itemize}
    \item paypal.controller.ts: Maneja captura de pagos y procesamiento de webhooks.
    \item configuracion.controller.ts: Gestiona configuración de límites y parámetros del sistema.
  \end{itemize}
  
  \item Modelos: Extensión de modelos existentes y nuevos esquemas:
  \begin{itemize}
    \item Usuario: Nuevos campos tipoUsuario (0=superadmin, 1=admin, 2=básico, 3=premium), fechaVencimientoPremium, limitePublicaciones.
    \item Publicacion: Campo pendingUpdate para almacenar cambios propuestos, editCount, lastEditRequest.
    \item Configuracion (nuevo): Esquema clave-valor para parámetros del sistema.
    \item Payment (nuevo): Auditoría de transacciones PayPal.
  \end{itemize}
  
  \item Rutas: Nuevos endpoints REST:
  \begin{itemize}
    \item POST /api/paypal/capture: Captura orden de PayPal y actualiza usuario.
    \item POST /api/paypal/webhook: Recibe eventos asíncronos de PayPal.
    \item GET /api/configuracion/mis-limites: Consulta límites de usuario autenticado.
    \item PUT /api/configuracion/limites-publicaciones: Actualiza límites globales (admin).
    \item PUT /api/publicaciones/admin/:id/approve-update: Aprueba cambios pendientes.
    \item PUT /api/publicaciones/admin/:id/reject-update: Rechaza cambios pendientes.
  \end{itemize}
  
  \item Middlewares: Extensión de capacidades:
  \begin{itemize}
    \item verificarLimitesPublicaciones: Middleware que valida límites antes de creación.
    \item verificarUsuarioPremium: Valida membresía activa y no vencida.
  \end{itemize}
  
  \item Utilidades: Nuevas utilidades especializadas:
  \begin{itemize}
    \item paypal.ts: Funciones para interactuar con PayPal API (getAccessToken, captureOrder, verifyWebhookSignature).
    \item paymentRetry.ts: Sistema de reintentos con backoff exponencial.
    \item paymentErrorHandler.ts: Clasificación y manejo de errores de pago.
  \end{itemize}
  
  \item Interfaces TypeScript: Contratos de datos:
  \begin{itemize}
    \item payment.interface.ts: Tipos para PaymentError, RetryHistoryEntry, PaymentResult.
    \item configuracion.interface.ts: Tipado para entidad de configuración.
  \end{itemize}
\end{itemize}

\subsection*{Arquitectura Frontend}

El frontend en React.js se enriqueció con componentes especializados para el flujo de membresías:

\begin{itemize}
  \item Componentes de membresía:
  \begin{itemize}
    \item CheckoutPremium.js: Página completa de checkout con integración PayPal SDK, selección de planes (mensual/anual), manejo de estados de pago (procesando, error, éxito) con animaciones visuales y reintentos.
    \item AlertaLimitePublicaciones.js: Modal que se activa cuando usuario básico alcanza su límite, mostrando estadísticas de uso con barras de progreso y botón CTA para upgrade.
    \item ModalLimitesPublicaciones.js: Panel administrativo para configurar límites globales de publicaciones para usuarios básicos y premium.
  \end{itemize}
  
  \item Componentes de moderación:
  \begin{itemize}
    \item Vista comparativa de cambios en perfilUsuario.js: Renderiza diferencias lado a lado con código tachado para valores eliminados y resaltado verde para nuevos.
    \item Sistema de pestañas: "Publicaciones Nuevas" vs "Actualizaciones Pendientes" para organizar tareas administrativas.
  \end{itemize}
  
  \item Integración PayPal:
  \begin{itemize}
    \item @paypal/react-paypal-js: SDK oficial de PayPal para React.
    \item PayPalScriptProvider: Contexto global para configuración de PayPal.
    \item PayPalButtons: Componente de botones con callbacks (createOrder, onApprove, onError, onCancel).
  \end{itemize}
  
  \item Gestión de estado:
  \begin{itemize}
    \item Variables de estado para tracking de proceso de pago: procesando, reintentos, errorMessage.
    \item Hook useEffect para cargar datos de límites del usuario autenticado.
    \item Estado local para pestañas administrativas y datos de actualizaciones pendientes.
  \end{itemize}
\end{itemize}

\subsection*{Comunicación y Seguridad}

El sistema de autenticación JWT se mantiene del Sprint 1, con adiciones de seguridad:

\begin{itemize}
  \item Verificación de webhooks: Validación criptográfica de firmas HMAC-SHA256 usando certificados de PayPal para garantizar autenticidad de eventos.
  \item Variables de entorno: Credenciales sensibles (PAYPAL_CLIENTID, PAYPAL_SECRET, PAYPAL_WEBHOOK_ID) almacenadas en .env y nunca expuestas en cliente.
  \item HTTPS obligatorio: PayPal requiere HTTPS para webhooks en producción, configurado en servidor.
  \item Idempotencia de pagos: Prevención de procesamiento duplicado mediante índices únicos en captureId y eventId.
\end{itemize}

\section{Diseño de la persistencia}

El modelo de persistencia se extendió para soportar membresías premium y configuración dinámica, manteniendo compatibilidad con el esquema del Sprint 1.

\subsection*{Modelo Conceptual Actualizado}

Cambios realizados en el Sprint 2:

\begin{itemize}
  \item Usuario: Se añadieron campos:
  \begin{itemize}
    \item tipoUsuario: Enum (0=superadmin, 1=admin, 2=básico, 3=premium).
    \item fechaVencimientoPremium: Date opcional para gestionar suscripciones.
    \item limitePublicaciones: Number opcional para overrides personalizados.
    \item publicacionesActuales: Contador calculado dinámicamente.
  \end{itemize}
  
  \item Publicacion: Se extendió con campos de moderación:
  \begin{itemize}
    \item pendingUpdate: Object que almacena cambios propuestos por usuarios premium.
    \item editCount: Number, contador de ediciones realizadas (máximo 3).
    \item lastEditRequest: Date, timestamp de última solicitud de edición.
  \end{itemize}
  
  \item Configuracion (nueva entidad):
  \begin{itemize}
    \item clave: String único (ej: "limite_publicaciones_basico").
    \item valor: Mixed (Number, String, Boolean según clave).
    \item descripcion: String para documentación interna.
    \item tipo: String ("number", "string", "boolean").
  \end{itemize}
  
  \item Payment (nueva entidad):
  \begin{itemize}
    \item captureId, eventId: String únicos para idempotencia.
    \item orderId: String, referencia a orden de PayPal.
    \item status: String (COMPLETED, APPROVED, FAILED).
    \item value, currency: Number y String para monto.
    \item userId: ObjectId referencia a Usuario.
    \item source: String ("capture" o "webhook").
    \item retryHistory: Array de intentos con timestamps.
    \item raw: Object con evento completo de PayPal.
  \end{itemize}
  
  \item Sin cambios: Categoria, Comentario, Archivo, Carpeta.
\end{itemize}

\begin{figure}[H]
    \centering
    \includegraphics[width=\textwidth]{project/images/Diagrama.jpeg} % Cambia la ruta según tu imagen
    \caption{Diagrama arquitectura}
    \label{fig:acerca-header}
\end{figure}

\subsection*{Justificación de los cambios}

\begin{enumerate}
  \item Separación de roles premium: El campo tipoUsuario con valor 3 permite diferenciar usuarios premium sin crear nueva entidad, simplificando queries y manteniendo consistencia.
  
  \item Versionado de publicaciones: El enfoque de pendingUpdate evita modificar publicaciones visibles directamente, permitiendo reversión fácil y auditoría completa de cambios.
  
  \item Configuración dinámica: La entidad Configuracion centraliza parámetros ajustables sin código, facilitando cambios operativos (ej: ajustar límites según demanda).
  
  \item Auditoría completa de pagos: La colección payments proporciona trazabilidad completa para soporte, contabilidad y debugging, incluyendo historial de reintentos.
  
  \item Idempotencia mediante índices: Los índices únicos en captureId y eventId garantizan que eventos duplicados de PayPal (common en webhooks) no causen double-charging o corrupción de datos.
\end{enumerate}

\subsection*{Estrategia de Almacenamiento}

Se mantiene la estrategia híbrida del Sprint 1:

\begin{itemize}
  \item Imágenes de publicaciones: Digital Ocean Spaces (CDN para rendimiento).
  \item Documentos de biblioteca: Almacenamiento local en servidor.
  \item Datos transaccionales: MongoDB para configuración, pagos y usuarios.
  \item Logs de errores de pago: Colección payment_errors (futura) para análisis de problemas recurrentes.
\end{itemize}

\section{Diagrama de clases}

El diagrama de clases refleja las extensiones del Sprint 2, incorporando clases de pago y configuración.

\subsection*{Clases principales}

\begin{itemize}
  \item Usuario: Atributos extendidos (tipoUsuario, fechaVencimientoPremium, limitePublicaciones). Métodos: isPremium(), canPublish(), getRemainingPublications().
  
  \item Publicacion: Nuevos atributos (pendingUpdate, editCount, lastEditRequest). Métodos: requestEdit(changes), approveEdit(), rejectEdit(), canEdit().
  
  \item Configuracion (nueva): Atributos (clave, valor, tipo, descripcion). Métodos estáticos: get(clave), set(clave, valor), getDefaults().
  
  \item Payment (nueva): Atributos (captureId, orderId, status, value, userId, retryHistory). Métodos: isIdempotent(), addRetry(attempt), markCompleted().
  
  \item PayPalService (nueva clase de servicio): Métodos: getAccessToken(), captureOrder(orderId), verifyWebhookSignature(headers, body), extractPaymentInfo(resource).
  
  \item PaymentRetryService (nueva): Métodos: retryWithExponentialBackoff(operation, maxAttempts), calculateDelay(attemptNumber), shouldRetry(error).
  
  \item Sin cambios: Categoria, Comentario, Archivo, Carpeta.
\end{itemize}

\subsection*{Relaciones nuevas}

\begin{itemize}
  \item Usuario 1---* Payment: Un usuario puede tener múltiples transacciones.
  \item Payment *---1 Usuario: Cada pago está asociado a un usuario (vía userId).
  \item Publicacion 1---0..1 PendingUpdate: Cada publicación puede tener máximo una actualización pendiente.
  \item Configuracion sin relaciones directas (singleton pattern).
\end{itemize}

[[INSERTAR IMAGEN: Diagrama de clases actualizado del Sprint 2 mostrando nuevas clases Payment, Configuracion, PayPalService y relaciones con Usuario y Publicacion]]

\subsection*{Patrones de diseño aplicados}

\begin{enumerate}
  \item Repository Pattern: Abstracción de acceso a datos en controladores.
  \item Service Layer Pattern: Lógica de negocio de pagos encapsulada en PayPalService y PaymentRetryService.
  \item Strategy Pattern: Diferentes estrategias de reintento según tipo de error.
  \item Observer Pattern: Webhooks de PayPal actúan como notificaciones asíncronas.
  \item Singleton Pattern: Configuración global única consultable desde cualquier módulo.
\end{enumerate}
