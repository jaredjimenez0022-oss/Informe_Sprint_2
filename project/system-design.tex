\chapter{Arquitectura del sistema}
\section{Diseño general del sistema}

Descripción en prosa del diseño del sistema. Resalte cualquier detalle que considere de importancia para la comprensión del diseño. Indique cuál patrón arquitectónico está utilizando como base para su sistema. Si está utilizando un \textit{framework} para apoyar el proceso de desarrollo, descríbalo en esta sección. Cite y referencie adecuadamente la información que lo requiera.

\subsection{Diseño de la persistencia}
Describa en esta sección la estrategia que abordó para diseñar la persistencia del sistema, incluya un diagrama para el modelo conceptual (Entidad-Relación) del sistema. Especifique cuáles aspectos del diseño fue necesario cambiar desde la última iteración, justifique todos los cambios.

\begin{figure}[H]
  \centering
    \includegraphics[height= 10cm, width=15cm]{project/images/data-sync}
  \caption{\textbf{IMAGE CAPTION}}
\end{figure}

Incluya y describa el modelo lógico (Diagrama relacional) de su sistema. Especifique todos los aspectos del diseño que fue necesario cambiar desde la última iteración, justifique todos los cambios.

\begin{figure}[H]
  \centering
    \includegraphics[height= 10cm, width=15cm]{project/images/data-sync}
  \caption{\textbf{IMAGE CAPTION}}
\end{figure}

\subsection{Diagrama de clases}
Añada una imagen con el diagrama de clases del sistema,  preferiblemente desarrollado en DIA. Introdúzcala con un párrafo breve que enlace el diagrama con la descripción en prosa del encabezado de sección. Se recomienda que el diagrama se diseñe de forma que calce bien en una página vertical, de forma que se facilite su lectura. Recuerde hacer referencia a su diagrama de clases usando la referencia correcta. Por ejemplo, en la Figura \ref{DiagramaClases} se encuentra el diagrama de clases.
\begin{figure}[H]
  \centering
    \includegraphics[ width=15cm]{project/images/class-example-online-shopping-domain.png}
  \caption{\textbf{IMAGE CAPTION}}
  \label{DiagramaClases}
\end{figure}


