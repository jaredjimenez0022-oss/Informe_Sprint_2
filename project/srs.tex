\chapter{Especificación de requerimientos}
\section{Pila de producto}
La plataforma digital para la comunidad de Tejarcillos continúa su desarrollo con el Sprint 2, enfocado en la monetización mediante un sistema de usuarios premium y pagos con PayPal. Para mantener la trazabilidad del sistema, se continúa con la codificación RFXXX (Requerimiento Funcional) y RNFXXX (Requerimiento No Funcional) establecida en el Sprint 1.



% La pila del producto debe actualizarla cada iteración del proyecto. Debe documentar los cambios sufridos durante la iteración pasada, de forma que se pueda trazar con mucho detalle los cambios que ha sufrido la definición del proyecto desde sus inicios. Si durante el desarrollo cambia alguno de los MVP o la EDT, debe modificarlos también y documentar la razón que justifica el cambio.

\subsection{Requerimientos funcionales del sistema}

%% PILA DE PRODUCTO %%
\begin{longtable}{|l||p{7cm}|l|l|}
\multicolumn{4}{c}{Pila general del producto - Sprint 2}\\
\hline\hline
\textbf{Código} & \textbf{Descripción} & \textbf{Prioridad} & \textbf{Inserción}\\
\hline
\endfirsthead
\textbf{Código} & \textbf{Descripción} & \textbf{Prioridad} & \textbf{Inserción}\\
\hline\hline
\endhead

\textbf{RF011} & 
\textbf{Corrección de filtro de búsqueda en biblioteca y límite de tamaño de archivos:} Reparar el comportamiento del filtro que mantiene resultados anteriores al borrar el texto de búsqueda, implementando reset adecuado del estado y actualización en tiempo real de los resultados. Solucionar error 413 Request Entity Too Large configurando el servidor para aceptar archivos hasta 100MB y mejorar mensajes de error al usuario. & Alta & Original \\


\textbf{RF012} & 
\textbf{Reparación del sistema de recuperación de contraseña en producción:} Corregir error 504 Gateway Timeout en el endpoint de recuperación, validar configuración de correos en producción y asegurar funcionamiento del flujo completo. & Alta & Original \\


\textbf{RF013} & 
\textbf{Extensión del modelo de usuario con campos premium e implementación de validación de límites de publicaciones:} Incorporar al campo tipoUsuario (3: premium), fechaVencimientoPremium, límitePublicaciones al esquema de usuario existente. Middleware que verifique los límites de publicaciones según tipo de usuario antes de permitir la creación de nuevas publicaciones & Alta & Original \\


\textbf{RF014} & 
\textbf{Configuración de SDK de PayPal en entorno sandbox:} Implementar autenticación y configuración inicial con credenciales de pruebas de PayPal, incluyendo setup de entorno de desarrollo. & Alta & Original \\


\textbf{RF015} & 
\textbf{Webhooks y procesamiento de pagos backend:} Endpoints para recibir confirmaciones de PayPal y actualizar automáticamente el estado del usuario a premium tras pago exitoso. & Alta & Original \\


\textbf{RF016} & 
\textbf{Sistema de reintentos y manejo de errores de pago:} Mecanismos robustos para manejar fallos de conexión, pagos rechazados, timeouts y otros escenarios de error en el flujo de pago. & Alta & Original \\


\textbf{RF017} & 
\textbf{Sistema completo de gestión de sección ``Acerca de`` e incorporación de nuevos campos en BD y Frontend:} Crear modelo en base de datos, API backend con CRUD completo e interfaz frontend interactiva con editor de texto enriquecido para la sección ``Acerca de" , incorporar campos adicionales en publicaciones (precio regular, precio estudiante/ciudadano de oro y enlaces externos), y corregir aspectos de UI/UX como visibilidad de contraseña en móviles y restricción de gestión de roles exclusivamente al rol 0 en el dashboard.& Baja & Original \\
 

\textbf{RF018} & 
\textbf{Sistema de alertas de límites en frontend e integración de las funcionalidades de membresía:} Componente React que muestre alertas cuando usuario básico alcance su límite de publicaciones y ofrezca opción de upgrade a premium.Y componente de checkout que guíe al usuario por el proceso completo de pago PayPal, desde selección de membresía hasta confirmación. & Alta & Original \\ 

\textbf{RF019} & 
\textbf{Sistema de moderación para ediciones de publicaciones premium:} Permitir a usuarios premium editar o borrar sus publicaciones, con notificación automática a administradores en la sección en panel de admin para revisar y aprobar/eliminar publicaciones actualizadas que puedan contener contenido inapropiado. & Alta & Original \\


\textbf{RF020} & 
\textbf{Pruebas integrales del sistema premium, PayPal y correción de errores:} Validación exhaustiva de todos los flujos del sistema premium e integración PayPal tanto en entorno de desarrollo como producción. Hacer correciones en caso de existir. & Media & Original \\


\textbf{RF021} & 
\textbf{Sistema de notificaciones por correo para moderación de publicaciones:} Implementar envío automático de correos a administradores cuando haya publicaciones pendientes de aprobación y notificaciones a usuarios cuando sus publicaciones sean aprobadas, reutilizando la infraestructura existente de Nodemailer con Zoho. & Media & Original \\

\textbf{RNF003} & 
\textbf{Manual de usuario:} Documentación completa de las funcionalidades premium y proceso de pago para usuarios finales y administradores. Además, agregar las actualizaciones realizadas en el sistema previo. & Media & Original \\


\textbf{RNF004} & 
\textbf{Manual técnico:} Documentación técnica detallada de la arquitectura del sistema premium, integración PayPal y APIs desarrolladas. Además, agregar las actualizaciones realizadas en el sistema previo. & Media & Original \\


\textbf{RNF005} & 
\textbf{Informe del Sprint 2:} Documentación del proceso de desarrollo. Además agregar las actualizaciones realizadas en el sistema previo. & Media & Original \\

\hline
\caption{\color{ForestGreen}VERDE: Historias agregadas en esta iteración. \color{Mahogany}ROJO: Historias eliminadas.}
\label{ProductBacklog}
\end{longtable}

\subsection{Bitácora de cambios}

Durante la planificación y ejecución del Sprint 2, no se realizaron cambios 



\subsubsection*{Historias modificadas}

Durante la  ejecución del Sprint 2, no se hubo historias modificadas  


\subsubsection*{Historias eliminadas}

No se eliminaron historias durante este sprint. Todas las historias planificadas fueron completadas exitosamente.

\subsubsection*{Decisiones técnicas relevantes}

\begin{itemize}
    \item Se decidió usar PayPal Sandbox en lugar de cuentas reales de producción durante todo el Sprint 2 para evitar riesgos financieros durante desarrollo y pruebas.
    
    \item Las imágenes de la sección Acerca de se almacenan localmente (/public/uploads) en lugar de Digital Ocean Spaces para simplificar implementación. Esto se migrará a CDN en sprints futuros.
    
    \item El sistema de reintentos (RF016) se implementó como utilidad reutilizable (paymentRetry.ts) en lugar de código específico de PayPal, permitiendo su uso en futuras integraciones de pago.
    
    \item Se validó con el cliente que el límite inicial de 5 publicaciones para básicos y 10 para premium era adecuado, basándose en patrones de uso observados en Sprint 1.
\end{itemize}

%% PILA DEL SPRINT 2 %%

%% PILA DEL SPRINT 2 %%

\subsection{Pila de trabajo de la iteración 2}

El equipo de desarrollo está compuesto por tres integrantes con dedicación de 12 horas semanales cada uno trabajando durante 4 semanas (20 días hábiles), totalizando 144 horas de capacidad al 100\%. Considerando un factor de productividad del 90\% para imprevistos y reuniones de coordinación, el presupuesto disponible es de 128 horas equivalentes a 32 puntos de historia para el \textbf{Sprint 2}.

\begin{longtable}{|l|l|l|}
    \hline
    \textbf{Miembro del equipo} & \textbf{Rol} & \textbf{Capacidad (puntos, 100\%)} \\
    \hline
    Fredrik Aburto Jiménez & Desarrollo Fullstack & 12 \\
    Angélica Díaz Barrios & Desarrollo Fullstack & 12 \\
    Andrés Salas Araya & Desarrollo Fullstack & 12 \\
    \hline\hline
    \textbf{Presupuesto total (100\%):} & & 36\\
    \textbf{Presupuesto disponible (90\%):} & & 32\\
    \textbf{Presupuesto en horas al 90\% (1 punto = 4 horas):} &  & 128\\
    \hline
    \caption{Resumen de capacidad de carga del equipo para la iteración 2}
\end{longtable}

\begin{longtable}{|l||c|c|p{7cm}|l|}
\multicolumn{5}{c}{Pila del \textbf{Sprint 2}} \\
\hline\hline
\textbf{Código} & \textbf{CE} & \textbf{CR} & \textbf{Responsables} & \textbf{Finalización} \\
\hline
\endfirsthead
\textbf{Código} & \textbf{CE} & \textbf{CR} & \textbf{Responsables} & \textbf{Finalización} \\
\hline\hline
\endhead

\textbf{RF011} & 1 & 1 & Fredrik Aburto Jiménez & 2025-10-02 \\
\textbf{RF012} & 1 & 1 & Andrés Salas Araya & 2025-10-01 \\
\textbf{RF013} & 2 & 2 & Fredrik Aburto Jiménez & 2025-10-08 \\
\textbf{RF014} & 1 & 1 & Andrés Salas Araya & 2025-10-03 \\
\textbf{RF015} & 3 & 4 & Andrés Salas Araya & 2025-10-16 \\
\textbf{RF016} & 3 & 3 & Fredrik Aburto Jiménez & 2025-10-23 \\
\textbf{RF017} & 3 & 3 & Angélica Díaz Barrios & 2025-10-13 \\
\textbf{RF018} & 3 & 3 & Fredrik Aburto Jiménez & 2025-10-17 \\
\textbf{RF019} & 3 & 3 & Angélica Díaz Barrios & 2025-10-20 \\
\textbf{RF020} & 3 & 3 & Angélica Díaz Barrios & 2025-10-23 \\
\textbf{RF021} & 2 & 2 & Andrés Salas Araya & 2025-10-19 \\
\textbf{RNF003} & 2 & 3 & Fredrik Aburto Jiménez & 2025-10-27 \\
\textbf{RNF004} & 3 & 3 & Andrés Salas Araya & 2025-10-27 \\
\textbf{RNF005} & 2 & 2 & Angélica Díaz Barrios & 2025-10-27 \\
\hline
\textbf{Total} & \textbf{32} & \textbf{34} & & \\
\hline

\caption{Pila de la Iteración 2. \textbf{CE:} Carga Estimada, \textbf{CR:} Carga Real.}
\label{SprintBacklog}

\end{longtable}

\subsection{Desempeño del equipo}

El equipo trabajó de manera colaborativa durante 4 semanas (del 30 de septiembre al 24 de octubre de 2025), completando exitosamente todas las historias planificadas para el Sprint 2.

\subsubsection*{Velocidad por integrante (CR)}

\begin{itemize}
    \item \textbf{Angélica Díaz Barrios:} 11 puntos
    \item \textbf{Fredrik Aburto Jiménez:} 12 puntos
    \item \textbf{Andrés Salas Araya:} 11 puntos
\end{itemize}

\subsubsection*{Problemas y desafíos enfrentados}

Durante el desarrollo se presentaron las siguientes situaciones que impactaron la ejecución:

\begin{itemize}
    \item \textbf{Complejidad subestimada en RF015 (Webhooks):} La historia de webhooks de PayPal fue estimada en 3 puntos pero requirió 4 puntos reales. La verificación de firmas de webhook resultó más compleja de lo previsto, requiriendo investigación adicional de la documentación de PayPal y ajustes en la configuración del servidor para recibir peticiones externas en desarrollo.
    
    \item \textbf{Scope creep en RNF003 (Manual de usuario):} El manual de usuario fue estimado en 2 puntos pero requirió 3. Durante la redacción se identificó la necesidad de incluir capturas de pantalla detalladas del flujo de pago completo y casos de error, lo cual no se consideró en la estimación inicial.
    
 
    \item \textbf{Dependencias bloqueantes:} RF018 (Alertas de límites) dependía de la finalización de RF013 (Modelo usuario premium). Esto generó un pequeño retraso de 2 días, pero se compensó con trabajo paralelo en RF017 (Sección Acerca de).
\end{itemize}

\subsubsection*{Cambios en la pila de trabajo}


\subsubsection*{Métricas del sprint}

\begin{itemize}
    \item \textbf{Velocidad del equipo:} 34 puntos completados en 4 semanas.
    \item \textbf{Carga estimada total (CE):} 32 puntos.
    \item \textbf{Carga real total (CR):} 34 puntos.
    \item \textbf{Desviación:} +6.25\% (2 punto adicional).
    \item \textbf{Tasa de completitud:} 100\% de historias finalizadas.
    \item \textbf{Horas trabajadas:} 136 horas (vs. 128 horas planificadas = 90\% del presupuesto total de 144 horas).
\end{itemize}

\bigskip

La siguiente gráfica de quemado  refleja el avance acumulado del sprint con respecto a la carga estimada:

\begin{table}[H]
\centering
\begin{tabular}{|c|c|c|}
\hline
\textbf{Día} & \textbf{Tendencia} & \textbf{Real} \\ \hline
0  & 32   & 34 \\ \hline
1  & 30,4 & 34 \\ \hline
2  & 28,8 & 34 \\ \hline
3  & 27,2 & 33 \\ \hline
4  & 25,6 & 32 \\ \hline
5  & 24   & 31 \\ \hline
6  & 22,4 & 31 \\ \hline
7  & 20,8 & 31 \\ \hline
8  & 19,2 & 29 \\ \hline
9  & 17,6 & 29 \\ \hline
10 & 16   & 29 \\ \hline
11 & 14,4 & 26 \\ \hline
12 & 12,8 & 26 \\ \hline
13 & 11,2 & 26 \\ \hline
14 & 9,6  & 22 \\ \hline
15 & 8    & 19 \\ \hline
16 & 6,4  & 14 \\ \hline
17 & 4,8  & 14 \\ \hline
18 & 3,2  & 14 \\ \hline
19 & 1,6  & 8  \\ \hline
20 & 0    & 0  \\ \hline
\end{tabular}
\caption{Tabla de referencia utilizada para el cálculo de la gráfica de quemado del Sprint 2.}
\label{tab:burndown_sprint1}
\end{table}
\subsubsection*{Gráfica de quemado}



% Gráfica de quemado
\begin{figure}[H]
  \centering
    \includegraphics[height=10cm, width=15cm]{project/images/grafica.png}.
  \caption{Gráfica de quemado del Sprint 2 mostrando progreso del equipo durante las 4 semanas de desarrollo}
  \label{fig:burndown-sprint2}
\end{figure}

\subsubsection*{Explicación de diferencias entre carga estimada y real}

La carga real (\textbf{34} puntos) superó ligeramente la carga estimada (\textbf{32} puntos) por las siguientes razones:



A pesar de la desviación del 6.25\%, el equipo completó todas las historias dentro del plazo establecido, manteniendo alta calidad en los entregables. 


